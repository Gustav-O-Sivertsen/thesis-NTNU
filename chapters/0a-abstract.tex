\chapter*{Abstract}

The traditional electric power grid, which delivers electric power to consumers, is in the process of being transformed from a system that is closed and centrally controlled to a network-controlled automated system for delivering electric power to consumers.
As part of this transformation, the control and monitoring subsystems of the classic power grid are transformed from being a location-based closed system, to an Internet-connected Smart Grid. In addition, network-connected devices are distributed that measure and adjust power consumption to the location of power consumers.

The transition described opens up the possibility that the power distribution infrastructure could become an offering for malicious cyber-attacks, with the potential intent of causing power outages, as well as serious damage to critical power grid infrastructure. The topic of my thesis is to investigate the potential effects of setting phasor measurement devices for time delay attacks, by exploiting known vulnerabilities in the IEEE 1588 Precision Time Protocol, observing any consequences such exposure may have on Synchrophasor measurements, as well as attack detection capabilities.

The master's thesis deals with security threats aimed at smart networks for the distribution of electrical energy. As part of the modernization of the power grid to meet future needs, the infrastructure has been connected to the Internet. This entails security challenges, in that malicious actors can carry out internet-based attacks on the infrastructure. In addition, the new infrastructure has an increased need for continuous monitoring, so that the requirement for precision related to monitoring events with a correct time stamp is a key to getting a correct overview of the system's condition. If actors can interfere with the mechanisms that calculate the time stamps of critical distribution components, the operators who monitor and control systems get the wrong decision basis, which can lead to an incorrect response, based on an incorrect basis. The time stamps are calculated based on time data, often obtained from clocks synchronized via GPS, but also via clock synchronization over the network-based PTP protocol. The theme of the master's thesis is the extent to which smart power distribution systems are vulnerable to the fact that targeted modification of the clock signal means that errors in time calculations create operational disruptions based on an incorrect understanding of the situation.

As part of the task, simulations are run in MATLAB and Simulink which show results that are predictable enough to be optimized in connection with further work.