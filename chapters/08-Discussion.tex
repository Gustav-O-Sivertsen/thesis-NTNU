\chapter{Discussion} \label{chap:Discussions}


%Here you should discuss all aspect of your thesis and project. How did the process work? Which choices did you make, and what did you learn from it? What were the pros and cons? What would you have done differently if you were to undertake the same project over again, both in terms of process and product? What are the societal consequences of your work?

\section{Introduction}


\subsection{Verification of model}

For verification purposes, results for the delay level of 0 is presented.  In this case, the original and delayed graphs should be identical.
\subsubsection{Comments on the result:}
The results of running the simulation with a delay of Zero produces a number of figures of pMU output consisting of indistinguishable signal graphs for the Native and Delayed output, with identical standard deviation, and no traces of red parts in the graphs showing degree of tolerance, for any \acrshort{pmu} output Component under study. The result may be regarded as indicative of a correctly working delay function, as well as the dualPMU being a correctly designed PMU subsystem. 




\section{Comments on various aspects related to the results}

%Discussion of results related to expected results, according to theoretical conclusions.
For the Instant delay attack of levels above two, a tendency of small variations for Delay Level three and Six seems to be visible.
\begin{itemize}

\item A delay of One alters the value for the \textbf{Angle} component of around $120^0$ compared to the value for the native, undelayed, \acrshort{pmu} output.
\item A delay of Two alters the value for the \textbf{Angle} component of around $-120^0$ compared to the value for the native, undelayed, \acrshort{pmu} output.
   \item A delay of Three alters the value for the \textbf{Angle} component a small number, like  $-6^0$ for the level of three, compared to the value for the native  \acrshort{pmu} output.
\end{itemize}

Even though no simulations for levels above six are performed, there seems to be a pattern related to the angular component value. This phenomena may possibly be related to the $120^0$ separation between the three phases of the \acrshort{pmu} input signal.\\ 


From my inspection of the figures presenting the results, it seems to be cases of a combination of results obtained during some of the instant delay attack simulations. The Step-Wise delay attack with a Delay level of Two, for instance, is initiated the same way as the Instant Delay attack of a Delay Level of one.\\ 

Regardless of the attack being Instant or Step-wise, the termination of the attacks involves an instant drop of delay level from the delay level specified, to Zero. Therefore, the finals second of the simulation using the step-wise attack should be the same as that of the corresponding Instant delay attack.  



\subsection{Similarity Requirements}

There seems to be a consistent tendency of graphs showing smaller distance between the delayed output and the native \acrshort{pmu} output, and the lower values for ranges of the diff variable being compared with the tolerance levels in the plots located in the lower part of each figure.