\chapter{Discussion} \label{chap:Discussions}


%Here you should discuss all aspect of your thesis and project. How did the process work? Which choices did you make, and what did you learn from it? What were the pros and cons? What would you have done differently if you were to undertake the same project over again, both in terms of process and product? What are the societal consequences of your work?

\section{Introduction}


\subsection{Verification of model}

For verification purposes, results for the delay level of 0 is presented.  In this case, the original and delayed graphs should be identical.
\subsubsection{Comments on the result:}
The results of running the simulation with a delay of Zero produces a number of figures of pMU output consisting of indistinguishable signal graphs for the Native and Delayed output, with identical standard deviation, and no traces of red parts in the graphs showing degree of tolerance, for any \acrshort{pmu} output Component under study. The result may be regarded as indicative of a correctly working delay function, as well as the dualPMU being a correctly designed PMU subsystem. 




\section{Comments on various aspects related to the results}

Discussion of results related to expected results, according to theoretical conclusions.





\subsection{Similarity Requirements}
In Figure \ref{fig:VoltageInstantDelayOne} (see above), for a similarity of 5\%, the attack detected  at level 1\% would stay undetected at level 5\%. 