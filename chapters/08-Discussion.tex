\chapter{Discussion} \label{chap:Discussions}


%Here you should discuss all aspect of your thesis and project. How did the process work? Which choices did you make, and what did you learn from it? What were the pros and cons? What would you have done differently if you were to undertake the same project over again, both in terms of process and product? What are the societal consequences of your work?

\section{Introduction}


\subsection{Verification of model}

For verification purposes, results for the delay level of 0 is presented.  In this case, the original and delayed graphs should be identical.




\section{Probable answers to the Research Questions}

Discussion of results related to expected results, according to theoretical conclusions.




As previously stated in the introductory chapter, the main goal of my thesis is to answer the following RQ's: 

\begin{enumerate}
    \item Which effects do the time delay attack have on PMU output?
    \item For a selected similarity requirement, what delay level could be within similarity tolerance levels?
    \item What would characterise the optimal delay function, for the malicious actor to stay undetected?    
    %\item How might a \acrshort{sp} attack be mitigated?
    %\item Investigate the GPS spoofing vulnerability of the \acrshort{sg} monitoring and control system.
    %\item Investigate GPS Spoofing detection and mitigation techniques. 
    %\item Investigate how \acrfull{sdn} might be applicable to improve \acrshort{sg} Security
    %\item Investigate SDN-based SG \acrshort{dos} detection and mitigation potentials. 
    %\item 
\end{enumerate}

During this chapter, I will analyse the results of the previous chapter, searcing for answers to the research questions.

\subsection{Effects of time delay attack on PMU output}

\subsubsection{PMU Magnitude output effects}
\subsubsection{PMU Angle output effects}
\subsubsection{PMU Frequency output effects}

\subsection{Similarity Requiremenst}
