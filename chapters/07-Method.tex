\chapter{Method} 

%The method chapter should describe in detail which activities you undertake to answer the research questions presented in the introduction, and why they were chosen. This includes detailed descriptions of experiments, surveys, computations, data analysis, statistical tests etc.

\section{Introduction}

As part of the introductory studies of the topic selected, a couple of searches on the NTNU literature search facilities returned a number of books, like \cite{BlumeStevenW2007Epsb}, \cite{kabalci2019smart}, and \cite{momoh2012smart}, covering introductory chapters on Power Grid and Smart Grid. The introductory chapters heavily relies on descriptions from relevant book chapters. \\ 


For the \acrfull{wams} Security parts, a small number of surveys, as well as a collection of papers are investigated in order to get positive evidence on what the community considers major threats to the operation of the \acrshort{sg} \acrshort{wams}. The introductory brief coverage of \acrshort{sg} security, is followed by a coverage of \acrlong{wams} security focusing on \acrlong{tsa}s. \\ 


During the phase concerning \acrlong{tsa} detection and mitigation, a selction of papers are covered as examples, in order to investigate possible techniques utilised in order to detect and mitigate the attacks.  \\ 


%In order to investigate, systems like mininet. \\ 


\section{Research Design}
%This section will inform the reader of the NATURE of your study. In other words, broadly speaking: are you aiming to describe a phenomena (descriptive design), are you aiming to explore a topic (exploratory design), are you looking to identify causal relationships between factors (causal design)?

The aim of the thesis, is to present an overview of concepts, as well as to to provide examples from the literature related to the Smart grid being vulnerable to GPS spoofing attacks. 

Based on the findings of the initial literature study concerning vulnerabilities, the literature study continues providing potential methods for GPS spoofing attack detection, and ultimately, mitigation.

As the final part of the thesis, a theoretical discussion, aiming to provide answers to the research questions, will be conducted.\\ 

\textbf{TO DO:}
\textit{As a experimental part of the thesis, possibilities of testing one or more detection and mitigation techniques might provide a contribution to the final discussion and conclusions based on my own experimental results}


\section{Research Methods}

%Following the description of your research design, you should also devote a section to describing the research methods you applied during your study. Each design will provide you with many possibilities of methods to use.

In order to provide theoretical evidence on which to answer the research questions, a literature study will be conducted.
For any experimental results, experiments will be described, implemented and executed.

\section{Measurements}

%Once you clarified the method you used, it is time to explain exactly WHAT you measured (e.g. service quality, brand image, satisfaction, purchase intention) and HOW you measured

Vulnerability to time-shift attacks are quantified by articles describing theoretical aspects of the mechanisms for the calculation of valid time-stamps, and the inherent tolerance level for calculation errors. Modern Smart Grids require the time deviating from the correct time by a fraction of a millisecond. Specifically, according to \cite[p.  1953]{moussa2016security}, the allowed time deviation for a 50Hz electrical system\footnote{As used in Norway, for instance} must be within $\pm$31.8 microseconds, in order to adhere to the 1$\%$ \acrfull{tve} requirement, as specified by the  IEEE C37.118 standard. \\ 

Experimental investigations reported by articles selected will be provided as relevant examples, in order to support any discussion arguments for the purpose of reaching conclusions.  

\section{Sample}

%In this section you should detail (at least!) the population of your study, your sampling technique (which technique you used to select the people who took part in your study) and how you established your sample size.
The samples for my literature study will be papers relevant for the discussions, in order to provide answers to the research questions.\\ 

Any execution of experiments will provide experimental samples, with the aim of supporting discussions and conclusions.
\section{Validity and Reliability}

%Now, here is a SUPER important section that 99\% get wrong(I completely made up this figure, simply because I want to convey a point!). Validity (that you measured what you intend to measure) and reliability (that the measurements used, such as your scales, are consistent and replicable) are two concepts that simply have to be addressed and have to do with your measurements.
In order to increase the validity and reliability, a number of articles will be included as the foundation for any conclusions. My personal selection of papers deemed relevant for my discussion, will be selected highlighting on articles being included as relevant articles by survey papers, as well as papers gaining a high relevance score on literature search sites.\footnote{... like the NTNU ORIA site (\url{https://innsida.ntnu.no/litteratur}).} Another selection criteria aiming to increase validity and reliability will be a focus on selecting articles receiving a high number of quotations ratings on sites like Google Scholar. %First and foremost, though, a sound and critical validation of the relevance for the questions at hand is still mandatory. 
\section{Infrastructure used during experiments }
\textbf{TO DO:}
\textit{A selection of a relevant infrastructure for experiments will be made according to  specific needs and availability.  }


%\section{Instruments or Equipment}

%Sometimes, especially in causal studies when researchers are developing experiments, it is important to detail the instruments or equipment that were used in the study.


%\subsection{The mininet environment}

%\subsubsection{Introduction to mininet}
%The experiments utilises the mininet environment which enables the construction of a software defined network for the purpose of providing a SDN-based test network.

%As described in the mininet documentation, mininet enables the definition of various network topologyes of hosts and switches, controoable by a selection of internal or eksternal SDN controllers.  


%\subsubsection{Prerequisites}
%\subsubsection{Description of mininet infrastructure used for testing}
%\subsubsection{Selected mininet Network topologies}

%In order to control the traffic flow of the mininet-based SDN, the OpenDaylight SDN controller is integrated with mininet in the test environment constructed for this study.


%\subsection{The openDaylight  SDN controller}
%\subsubsection{Introduction to OpenDaylight}
%\subsubsection{Prerequisites}
%\subsubsection{Configurations used}


\section{Experimental Procedure}
\textbf{TO DO:} \textit{Describe experimental procedure, dependant on experiments and experimental environment selected.}.
%Once again, in case you are running a causal study and an experiment, it is important to detail the experimental procedure.

%Explain, to the reader for example, what was the experimental task (what did the participants have to do?), the extraneous variables that were controlled (variables of the environment that could affect the cause and affect relationship).


