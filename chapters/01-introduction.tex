
\chapter{Introduction and Scope} 

%[Introduction:] The introduction would serve the same purpose as for a smaller research project described in \cref{sec:development}, but would normally be somewhat more extensive. The \emph{agenda} part should inform the reader about the structure of the rest of the document, since this may vary significantly between theses.

During the last few centuries, the society  has become increasingly dependant on access to a stable and reliant supply of electrical power. The usage of electrical power in order to provide heating, lighting, as well as running various electrical appliances, like washing machines and wireless access points, has created an increased demand for electricity. The increased demand for electricity has transformed the classical \acrfull{pg} infrastructure, initially consisting of local electricity distribution infrastructure,  into a nation-wide power distribution infrastructure network, known as the \acrfull{sg}.







\section{Background}
As described in \cite{BlumeStevenW2007Epsb}, electricity is supplied by the Power Grid infrastructure, consisting of power generating facilities,\footnote{like hydro-based facilities utilising water to generate electricity, or nuclear power plants.} and transmitted via a high voltage transmission infrastructure, before being converted to  lower voltage electrical currents, fed to the distribution infrastructure for distribution to paying consumers.


The classic \acrlong{pg} was under manual and centralised control, and monitored through a centralised and unidirectional \acrfull{scada} system.
The progressively increasing dependency on a reliable supply of electrical power makes it vital to avoid power outages, which will result in reduced temperature in buildings lacking alternative sources of heating, as well as the absence of vital services, like electronic payments, telecommunications, and transportation.
%As described in \cite{rehmani2019software}, the traditional \acrlong{pg} has proven to be unable to meet the demands of the future for a flexible and reliable power distribution system.
Therefore, several organisations, like the \acrfull{nist} and the \acrfull{ieee}\footnote{amongst others.} has made considerable efforts in order to produce standards, in order to define the \acrlong{sg}.


\subsection{The conversion of the Power Grid to the Smart Grid.}



%As described in \cite{greer2014nist} and  \cite{MRABET2018469}, the  \acrlong{sg} consists of seven domains, defined as presented in table \ref{tab:SmartGRID-Roles-of-domains}

In order to address the future demands for electrical power, the uni-directional control mechanisms of the traditional power grid is replaced by the bidirectional flow of information characteristic of the modern \acrshort{sg}. The control mechanisms of the modern \acrshort{sg} infrastructure is utilising standardised computer networking technology for bidrectional communication.  In order to meet the new requirements for monitoring, the \acrshort{pg} is connected to the Internet. 
The authors of \cite{colak2020effects}  describes the transition from the classical power grid to the smart grid as unavoidable.
%The \acrfull{sg} is the modern improvement of the classical \acrfull{pg}, meeting the increased demand for a more reliable, stable and flexible power distribution infrastructure.

For the system to meet the \acrlong{sg} monitoring  and control capability requirements, the \acrfull{wams} has been implemented. 
A number of \acrlong{pmu}s are connected to a \acrlong{pdc}, which transmits synchronised phasors to the \acrshort{wams} Control Center. The usage of the resulting synchrophasors enables the \acrshort{sg} operators to get a more fine-grained view of the operational state of the \acrshort{sg}, than obtainable by a classic \acrshort{scada} system. As part of a modern \acrshort{wams} system, \acrfull{se} algorithms has been developed, in order to counter cyber attacks or unintentional \acrshort{pg} failures. In addition to the \acrshort{se} algorithms, several security enhancements to relevant \acrshort{sg} protocols are being implemented in an attempt to reduce the attack surface available to potential threat actors.







\subsection{Security vulnerabilities of the Smart Grid WAMS}
As a consequence of the \acrlong{sg} infrastructure being connected to the Internet, the infrastructure is becoming vulnerable to cyber attacks.   The \acrlong{sg} \acrlong{wams} is the main control system of the \acrshort{sg} and, as such, constitutes a complex system vulnerable to numerous cyber attacks, as identified by several papers,  like for instance \cite{li2019review} and  \cite{kateb2018enhancing}.

The modern interconnected \acrshort{sg}, featuring a \acrshort{wams} system receiving an increasing number of synchronised phasors,\footnote{Phasors are synchronised at \acrshort{pmu}s, before being transferred via \acrshort{pdc}s to the WAMS,}  is increasingly dependant on synchronised time.
%As explained by \cite{xue2021data}, a \acrlong{tsa} might be successfully completed without prior knowledge of the infrastructure being targeted. \\ 
%Given the vulnerability of the \acrshort{gnss} to \acrshort{tsa}, the ever increasing likelihood of one or more of the \acrshort{pmu}s of the \acrshort{sg} \acrshort{wams} being the target of a \acrshort{tsa} causing any kind of havoc to the grid, makes this topic vital for further studies.
In \cite{ullmann2009delay}, several types of cyber attacks are identified. Most of the attacks described by \citeauthor{ullmann2009delay}in \cite{ullmann2009delay}, with the exception of the "Delay of synchronization messages" attack are, to some extent, being mitigated by various means. The 
Delay  synchronization messages attack however, may\footnote{if performed by a sophisticated threat actor having sufficient knowledge of valid traffic patterns.} prove to be a stealthy attack, thereby avoiding detection.  The aim of these sophisticated threat actors would, therefore, be to stay undetected, while creating a substantial amount of disturbance to the operation of the \acrshort{sg}. \\ 
% Given the aforementioned tendency of going online with components essential to the proper control of the \acrshort{sg}, opens the possibilities for attackers to  interfere with, for instance,  unencrypted synchrophasor data transmission from \acrlong{pmu}s to their corresponding \acrlong{pdc}s, by exposing them to a \acrlong{mitm} attack.
%A presentation of various attack strategies are given by \citeauthor{li2019review} in \cite{li2019review}, as well as by \citeauthor{paudel2017attack}in \cite{paudel2017attack}.
%The scope of my master thesis, will be to investigate vulnerabilities of \acrfull{tsa}, targeting the \acrfull{wams} in order to disturb proper \acrshort{sg} operation.
%The scope of my master thesis, will be to investigate vulnerabilities of the synchrophasor protocols, targeting the \acrfull{wams} in order to disturb proper \acrshort{sg} operation.
My thesis will be focusing on time delay attacks against \acrshort{pmu}s, and investigate which effects a number of selected delay attacks may have on the \acrshort{pmu} output, as visualised by a simulation. 


\section{Definition of scope}

%The introduction of Synchrophasors have enabled the \acrshort{sg} operators to gain increased\footnote{as compared to the traditional SCADA system} control over the operational state of the modern \acrlong{pg}. 
%As described by  \cite{ullmann2009delay}, the standardisation of Synchrophasor data transmissions will, with the introduction of transmission encryption, impose increasingly higher levels of security to the \acrlong{sg}. 
%As further described by \cite{ullmann2009delay}, the \acrlong{tda} still remains a threat to the \acrlong{ptp}.% which, according to \cite{bishop2022iec}, is identified as the preferred standard to be used for Time Synchronisation  for \acrshort{sg}s operating according to the IEC 61850 standard.


As \citeauthor{moussa2016security}denotes in TABLE II of \cite[p. 1959]{moussa2016security}, the paper \cite{ullmann2009delay}\footnote{ The paper, entitled "Delay attacks — Implication on NTP and PTP Time Synchronization",\\ is  identifiable as reference [13] in \cite{moussa2016security}.} by \citeauthor{ullmann2009delay}, is lacking experimental verification of the attack, the countermeasures  
suggested, \textit{as well as any effects of the \acrshort{ptp} delay attack}. 

%On the other hand, the quantitative study of \cite{ullmann2009delay}, is evaluating the consequences of the \acrshort{ptp} delay attack.\footnote{ as identified by the \textbf{Advantage} column of \textbf{TABLE II} of \cite[p. 1959]{moussa2016security}. }\\ 
Given the unresolved \acrshort{tda} vulnerability of the \acrshort{ptp}, in addition to the absence of a description of any effects of the attack, identified by \cite{moussa2016security} as a disadvantage of \cite{ullmann2009delay}, my contribution will, specifically, be to investigate possible effects of exposing a \acrlong{pmu} to a \acrlong{tda}. 
%The simulation is performed by producing a new signal by shifting the original signal by a number of steps, according to the formula $y(timeStamp)= x(timeStamp + delay)$.
%Given an attack stealthiness requirement, which delay level produces a result within the graph signal similarity tolerance level specified?
%In order to limit the scope of my investigation, two categories of attacks are specified:
%\begin{enumerate}
%    \item An attack by which the delay level is 0 before the attack, and increasing instantly to the specified level is reached.
%    \item An attack by which the delay level is 0 before the attack, and increasing one sample each second until the specified level. 
%    
%\end{enumerate}
%In each case, the attack is initiated at a specified sample time, and terminated at a specified time during the time span of the simulation. The termination of a real Time Delay attack could be intentional\footnote{Like the simulated controlled attack termination initiated by the threat actor}, or caused by a genuine synchronisation of the PTP clock under attack.
\section{Research questions}
The main goal of my project, is thus to investigate which potential effects a selection of \acrlong{tda}s might have on \acrshort{pmu} output. 

In order to proceed with the \acrlong{tda} vulnerability investigation, I have selected the following research questions:

\begin{enumerate}
    \item Which effects of the time delay attack simulations covered by this study, is observable on the visualised output of the PMU simulated?
    \item For a selected similarity requirement, what delay level could be observed as being within similarity tolerance levels?
    \item Which of the delay functions covered would be preferred, in order for the malicious threat actor to stay undetected?    

    %\item How might a \acrshort{sp} attack be mitigated?
    %\item Investigate the GPS spoofing vulnerability of the \acrshort{sg} monitoring and control system.
    %\item Investigate GPS Spoofing detection and mitigation techniques. 
    %\item Investigate how \acrfull{sdn} might be applicable to improve \acrshort{sg} Security
    %\item Investigate SDN-based SG \acrshort{dos} detection and mitigation potentials. 
    %\item 
\end{enumerate}

In order to investigate the topic further, I will conduct a theoretical study of relevant concepts, before performing a number of experiments before discussing the results presented, with the aim of being able to conclude on possible answers to the research questions. 



\section{Outline of the rest of the thesis}

%Following the introductory study of the concept of the \acrshort{sg}, I will study relevant protocols, before investigating the nature of a number of  time delay attacks.


This introductory chapter is followed by a theoretical part, consisting of a chapter introducing the (smart) power grid, followed by chapters covering topics like the \acrshort{wams} control system, \acrfull{sg} power flows status monitoring, and \acrlong{sg} cyber attacks and threat actors.  

The remaining chapters covers the experimental investigation of the \acrlong{tda} on \acrlong{pmu} output. The methodology chapter includes a detailed description of the model implementation and usage, before a number of experiments are presented and described in detail.
My thesis is finalised by the inclusion of a chapter presenting results, before the discussion and conclusion chapters, including the suggestions for further work, at the end of the thesis. \\


To summarise the content of each of the remaining Chapters:
\begin{itemize}
    \item Chapter2 presents a brief introduction to Power grids, focusing on the transition from the \acrlong{cpg} to the \acrlong{sg}.
    \item Chapter 3 presents the power grid control system, known as the \acrfull{wams}
    \item Chapter 4 Considers the status flow monitoring of th \acrlong{sg}, covering \acrshort{wams} benefits and security issues.
    \item Chapter 5 Considers attacks, mainly the \acrshort{ptp} \acrlong{tda}.
    \item Chapter 6 Provides the methodological background for the thesis, as well as a description of the model used for running the simulations.
    \item Chapter 7 Presents and interprets the results
    \item Chapter 8 Discusses the observed tendencies visible by the results.
    \item Chapter 9 Finalises, with Conclutions, and suggestions for future work.
\end{itemize}
 
