\chapter*{Sammendrag}

Det tradisjonelle elektriske kraftnettet, som leverer elektrisk kraft til konsumenter, er i ferd med å bli transformert fra et system som er lukket og sentralstyrt et nettverksstyrt automatisert system for å levere elektrisk kraft til forbrukere.
Som en del av denne transformasjonen transformeres kontroll- og overvåkingsundersystemene til det klassiske strømnettet fra å være et lokasjonsbasert lukket system, til et Internett-tilkoblet Smart Grid. I tillegg distribueres nettverkstilkoblede enheter som måler og justerer strømforbruket til strømforbrukernes plassering.

Overgangen som beskrives åpner muligheten for at kraftdistribusjonsinfrastrukturen kan bli et tilbud for ondsinnede cyberangrep, med den potensielle hensikten å forårsake strømbrudd, samt alvorlig skade på kritisk kraftnettinfrastruktur. Temaet for oppgaven min er å undersøke potensielle effekter av å sette faseormåleenheter for tidsforsinkelsesangrep, ved å utnytte kjente sårbarheter i IEEE 1588 Precision Time Protocol, observere eventuelle konsekvenser slik eksponering kan ha på Synchrophasor-målinger, samt angrepsdeteksjonsevne.

Masteroppgaven omhandler sikkerhetstrusler rettet mot smart nettverk for distribusjon av elektrisk energi. Som en del av moderniseringen av strømnettet for å møte framtidige behov, har infrastrukturen blitt tilknyttet Internett. Dette medfører sikkerhetsmessige utfordringer, ved at ondsinnede aktører kan utføre internettbaserte angrep på infrastrukturen. I tillegg har den nye infrastrukturen et økt behov for kontinuerlig overvåking, slik at kravet til presisjon relatert til åvå hendelser med korrekt tidsstempel er en nøkkel for å få en korrekt oversikt over systemets tilstand. Dersom aktører kan forstyrre mekanismene som beregner tidsstemplene til kritiske distribusjonskomponeneter, får operatørene som overvåker og styrer systemer feil beslutningsgrunnlag, noe som kan føre til feilaktig respons, basert på feilaktig grunnlag. Tidsstemplene blir beregnet basert på tidsdata, ofte hentet fra klokker synkronisert via GPS, men også via klokkesynkronisering over den nettverksbaserte PTP-protokollen. Masteroppgavens tema er i hvilken grad smarte kraftdistribusjonssystemer er sårbare for at målrettet modifisering av klokkesignalet medfører at feil i beregninger av tid skaper driftsforstyrrelser basert på en feilaktig situasjonsforståelse.

Som en del av oppgaven kjøres simularinger i MATLAB og Simulink som viser resultater som er forutsigbare nok til å kunne optimaliseres i forbindelse med  videre arbeid.
