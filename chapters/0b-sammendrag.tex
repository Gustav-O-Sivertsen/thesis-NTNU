\chapter*{Sammendrag}

Masteroppgaven omhandler sikkerhetstrusler rettet mot smarte nettverk for distribusjon av elektrisk energi. Den tradisjonelle infrastrukturen for distribusjon av elektrisitet  kjennetegnes av sentralstyring og en distribusjon av elektrisistet som ikke er påvirket av abonnentenes faktiske forbruksmønster, men tilpasset estimert behov. Som et svar på de opplevde utfordringer som tradisjonelle distribusjonssystemer har vist seg å ha hatt når det gjelder å sikre en stabilstrøforsyning tilpasset kundenes individuelle behov, har infrastrukturen blitt modernisert. Som en del av moderniseringen, har  infrastrukturen blitt forbundet med Internett, med de utfordringer dette medøfrer relatert til informasjonssikkerhet, med økt fare for driftsavbrugg forårsaket av at ondsinnede aktører utfører tjenestenetkangrap på infrastrukturen. I og med at den moderniserte infrastrukturen har behov for kontinuerlig overvåking av distribusjonssystemet, er kravet til presisjon relatert til å merke hendelser med korrekt tidsstempel avgjørende for å få det korrekte bildet av distribusjonssystemets tilstand. Dersom aktører kan forstyrre mekanismene som beregner tidsstemplene til kritiske distribusjonskomponeneter, får operatørene som overvåker og styrer systemene feil beslitningsgrunnlag, som dermed kan trigge responser basert på feilaktig grunnlag. Tidsstemplene blir beregnet basert på tidsdata hentet fra satelittbaserte tjenester som det amerikanske GPS-systemet. Masteroppgavens tema er i hvilken grad smarte kraftdistribusjonssystemer er sårbare for målrettet modifisering av GPS-signalet slik at feil i tidsberegninger skaper planlagte driftsforstyrrelser.
Som en videreføring av temaet, vil oppgaven deretter fokusere på hvilke muligheter det kan finnes for deteksjon av slike hendelser, med tanke på å hindre at angrepene ikke får sin tilsktede effekt.