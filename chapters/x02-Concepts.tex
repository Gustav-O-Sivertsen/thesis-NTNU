
 
 %Research projects should always be based on previous research on the same and/or related topics. This should be described as a background to the thesis with adequate bibliographical references. If the material needed is too voluminous to fit nicely in the review part of the introduction, it can be presented in a separate background chapter.
In order to get a theoretical overview of topics related to the subject of my thesis, the concept of \acrfull{sg} is explained, as well as aspects related to the \acrshort{sg} architecture, and \acrshort{sg} vulnerabilities.\\  

However, before studying the \acrshort{sg} in more detail,  some introductory terms are covered.  



\section{Critical (Information) Infrastructure}

The term Critical Infrastructure, denotes infrastructure critical to the operation of some function, or service, vital to its recipients. As a result of the society being more dependant on the availability of electronic services, the flow of electricity from the \acrshort{pg} \acrfull{ci} is dependant on networking provided by \acrfull{cii}.
\textbf{\acrfull{ci}} is, in  \cite{luiijf2012understanding}, defined as ... 
 \begin{quote}
"... an asset, system or part thereof located in a
nation which is essential for the maintenance of vital societal functions, health, safety,
security, economic or social well-being of people, and the disruption or destruction of
which would have a significant impact in that nation as a result of the failure to maintain
those functions." \cite[p. 53]{luiijf2012understanding}     
 \end{quote}
 

\textbf{\acrfull{cii}} is, in  \cite{luiijf2012understanding}, defined as ...
 \begin{quote}
" ... those interconnected information
systems and networks, the disruption or destruction of which would have a
serious impact on the health, safety, security, or economic well-being of citizens, or
on the effective functioning of government or the economy." \cite[p. 53]{luiijf2012understanding}     
 \end{quote}


The \acrshort{cii} is fundamental to the proper operation of the \acrshort{ci}, and is dependant on the availaibility of \acrfull{ict} and \acrfull{telco} \acrlong{ci}, as 
indicated by figure \ref{fig:CI-CIII}


\begin{figure}[ht]
\includegraphics[width=\linewidth]{figures/CI-CII.png}
\caption[CI/CIII in the ICT and TELCO sectors]{Sector CI and CII dependency on ICT and TELCO CI. \cite{luiijf2016gfce}}
\label{fig:CI-CIII}
\end{figure}


\subsection{Critical (Information) Infrastructure Protection}

As presented in figure \ref{fig:CI-CIII}, \acrfull{ci} depends on \acrfull{cii}, as \acrshort{cii} is required in order for a distributed \acrshort{ci} like the \acrshort{sg} to operate as required. Figure \ref{fig:CI-CIII-Security} shows the relationship between Critical Infrastructure Protection and Critical Information Infrastructure Protection, as the cyber security strategies related to \acrshort{ci} protection.


\begin{figure}[ht]
\includegraphics[width=\linewidth]{figures/CIP-CIIP-Security.png}
\caption[CIP-CIIP relations]{Relationship between Critical Infrastructure Protection and Critical Information Infrastructure Protection. \cite{luiijf2016gfce}}
\label{fig:CI-CIII-Security}
\end{figure}












\section{Cyber-Physical System}

As described by \citeauthor{humayed2017cyber} in \cite{humayed2017cyber}, a \acrfull{cps} is characterised by using a computer-based system in order to control and monitor systems of the physical world. The \acrshort{cps} is utilised in order to control physical systems from various fields of applications. Humayed et. al., in \cite{humayed2017cyber}, describes as various appliances as \acrfull{ics}, Medical Devices, and \acrlong{sg}.
The Smart Grid, therefore, is an example of a \acrfull{cps}, consisting of the physical system of a Power Grid, under the control of a network/Cyberspace-connected system. 
The \acrshort{sg} is, as described by several papers, like \cite{humayed2017cyber}% \#NEWREF and \cite{alcaraz2012security}
, under the control of a \acrfull{scada} system.




\










\subsection{Smart Grid as Critical Infrastructure}



As described in \cite{colesniuc2013cyberspace}, information technology is critical to the successful operation of the modern society.




 
 The EU Commission defines, in \cite{eu2008council}, \textit{National Critical Infrastructure}, to mean...:
 
 \begin{quote}
    "... an asset, system or part thereof
located in Member States which is essential for the maintenance of vital societal functions, health, safety, security,
economic or social well-being of people, and the disruption
or destruction of which would have a significant impact in a
Member State as a result of the failure to maintain those
functions"   \cite[p.  L 345/77]{eu2008council}  
 \end{quote}
 
 The disruption of an \textit{European Critical infrastructure}, according to  \cite{eu2008council}... \\ "...would have a significant impact of at least Two member states."

Recognised as critical infrastructure, the availability of the \acrlong{sg} is of paramount importance to the society. 



\subsection{Time Synchronisation Sources}

As described by \citeauthor{moussa2016security} in \cite{moussa2016security}, both PTP and GNSS is able to achieve the precision level required, as opposed to NTP and PPS.  Based on this observation, the \acrshort{tsa} Detection and Mitigation coverage is limited to \acrshort{gnss} and  \acrshort{ptp} systems only.




\subsubsection{Global Navigation Satellite Systems}

A \acrfull{gnss}, is a system of communication satellites orbiting the Earth, in order to determine the current position for, usually,\footnote{\acrlong{sg} systems utilises \acrshort{gnss} systems for time synchronisation, not navigational, purposes.} navigation  purposes. As described by \citeauthor{schmidt2016survey} in \cite{schmidt2016survey}, there exists several \acrshort{gnss} constellations,\footnote{In \cite{schmidt2016survey}, both the United States "GPS" and the Russian "GLONASS" systems, as well as the European "Galileo" and the Chinese "Beidou-2" systems are mentioned.} each being classified as "strikingly similar" to the other \acrshort{gnss} systems. 



%Therefore, any description of \acrshort{gnss} systems, as well as attacks targeting them, will be considering the well-known \acrfull{gps} system only.





As described by \citeauthor{schmidt2016survey} in  \cite{schmidt2016survey},  a \acrshort{gnss} system transmits three different messages:

\begin{enumerate}
    \item A PVT ranging signal for Position, Velocity, and Timing.
    \item Precise Ephemeris\footnote{Ephemeris is calculated based on prior observations of astronomical object trajectories} data, is specifying the exact location of each satellite.
    \item An almanac used in order to select satellites for tracking, based on the known location of all satellites. 
    
\end{enumerate}

$$r(t)=\sum_{k=1}^{32}H_{k}(2P_{c})^{\frac{1}{2}}(C_{k}(t)\oplus D_{k}(t))\cos{2\pi}(f_{L1}+\Delta f_{k})t+n(t) $$


As further described by \citeauthor{schmidt2016survey}, satellites from any other \acrshort{gnss} constellation may be used in order to avoid problems related to poor reception or blackout which would, possibly be the result of using same constellation satellites only.


The \acrshort{gnss} signal is utilised in order to provide time synchronisation between \acrshort{sg} \acrshort{pmu} devices, in order to enable the \acrshort{wams} system operators to get the best available 
view of \acrshort{sg} distribution system state.


\subsubsection{Description of GSSN time synchronisation}
The time signal available from \acrfull{gnss} systems is by several papers described as the major contributor to the wide-spread usage of \acrshort{pmu}s in order to monitor \acrlong{sg} energy flow status. 







The GPS signal transmits timing information, which enables 


$$t_{UTC}=t_{rcv}-t_{p}-\Delta t_{UTC}$$
A synchrophasor is, as described by 



%\textbf{TO DO:}
%\textit{Introduction:}

%\textit{What is it?}
%\textit{Which functions does it have?}





%Explain TimeSync 








A periodic signal $x(t)=X_m\cos\left(\omega t+\phi \right)$ has, according to \Cite{schofield2018design}, a synchrophasor representation $\mathbf{X}$, defined by $ \mathbf{X} = \frac{X_m}{\sqrt{2}}e^{j\phi} $ where $ \mathbf{X} = |X| = X_m / \sqrt{2} $ is  the RMS and $  \phi = angle(\mathbf{X}) $ is the angle.


%\[x(t)=X_m\cdot\cos\left(\phi + \int\displaylimits_{-\infty}^t\omega (\tau)\cdot\mathrm{d}\tau\right)\]

%\[\overline{X} = X_m{\angle\phi} \]








\section{cyber attacks}
\subsection{Introduction}
\subsection{type of attacks}
\subsection{Threat Actors}
\subsubsection{Introduction}
\subsubsection{Type of Threat Actors}
\subsection{Attack strategies}












As described in \cite{gopstein2021nist}, the following may constitute a definition of the \acrlong{sg}:

\begin{displayquote}[{\cite[p 27]{gopstein2021nist}}]
''A SmartGrid is an electricity network that can intelligently integrate the actions of all
users connected to it - generators, consumers and those that do both – in order to
efficiently deliver sustainable, economic and secure electricity supplies.
A SmartGrid employs innovative products and services together with intelligent monitoring,
control, communication, and self-healing technologies to:
\begin{itemize}
\item better facilitate the connection and operation of generators of all sizes and technologies;
\item allow consumers to play a part in optimizing the operation of the system;
\item provide consumers with greater information and choice of supply;
\item significantly reduce the environmental impact of the whole electricity supply system;
\item deliver enhanced levels of reliability and security of supply.
\end{itemize}
\acrshort{sg} deployment must include not only technology, market and commercial
considerations, environmental impact, regulatory framework, standardization usage, ICT
(Information \& Communication Technology) and migration strategy but also societal
requirements and governmental edicts. 
'' 
\end{displayquote}


The \acrshort{sg} is a part of the Critical Infrastructure of a Society, delivering a constant and reliable flow of stable and electrical power to consumers according to demand, in ways both cost efficient and friendly to the environment. The \acrshort{sg} consists of a modernised \acrshort{pg}, under the control of a network based control system. 


\textit{DESCRIPTION:}
\textbf{\cite{kumar2015monitoring} \fullcite{kumar2015monitoring} }


The \acrshort{sg} consists of seven domains, as shown in \figureautorefname { }\ref{fig:NIST-SmartGRID-ConceptualModel}:
\begin{itemize}
    \item \textbf{Customer Domain} The customers are the Consumers of Electricity.
    The power infrastructure of commercial or private customers includes \acrfull{ami}, monitoring the amount of energy consumed, both for billing and \acrfull{dr} purposes. Consumers may plan their consumption, avoiding high-cost periods of heavy load, by  selecting time frames of low prices.
    \item \textbf{Markets Domain} The participants of the Markets Domain aims to balance the consumption and demand of electricity, by adjusting prices on electricity. Price adjustments may be used in order to shift consumption from periods of high demand, to periods of low demand.     
    \item \textbf{Service Provider Domain} Services to the Customers,  as well as the Markets and Operators domain, are provided by the Service Provider Domain, fulfilling duties like customer management and billing, as well as a number of emerging services as required. 
    \item \textbf{Operations Domain} This domain consists of Electricity service operators, ensuring efficient and fail-safe \acrfull{sg} operation, by utilising \acrshort{scada} systems and \acrlong{ems}s in order to monitor and control system operational state.  
    \item \textbf{Bulk Generation Domain} The facilities for producing electricity, resides in this domain. In addition to the connection and interaction with  to the Transmission domain, it interacts with the Markets domain, as well as the operations domain.  
    \item \textbf{Transmission Domain} The actors of the Transmission domain aims to reduce energy loss while transmitting a stable and reliable stream of energy from operators in the bulk generation domain to the distribution domain. The market domain provides input on expected level of demand which may require adjustments of the amount of electricity distributed, controlled and monitored by actors in the operation domain.  
    \item \textbf{Distribution Domain} The actors of the Distribution Domain delivers the electricity to consumers according to demand and availability, and monitors generation and consumption data. Bi-directional power-flow is supported. In the case customers have private  power producing facilities, like solar cells and wind turbines, any surplus electricity might be sold, and distributed to other customers.
\end{itemize}




\begin{itemize}
    \item The \textbf{Generation} domain:
    \item The \textbf{Transmission} domain:
    \item The \textbf{Distribution} domain:
    \item The \textbf{DER} domain:
    \item The \textbf{Customer Premise} domain:
\end{itemize}


\subsection{Introduction}


The dependency on networking favors the usage of low-cost \acrshort{gnss} receivers for synchronising time between the growing number of \acrshort{pmu}s deployed at various locations, monitoring energy flow states of the highly distributed \acrshort{sg}. 


\subsection{Characteristics of the Smart Grid }


The transition of the \acrshort{cpg} into the \acrshort{sg}, has transformed the \acrshort{cpg} into a \acrshort{pg} having, according to \hl{REF!!}, the following characteristics:

\begin{itemize}
    \item The system consists of a physical system controlled by, and accessible from, a networked system known as the Cyber System.
    \item The interactions between the various systems are bidirectional, as opposed to the unidirectional \acrshort{cpg}.
    \item 
\end{itemize}


\subsection{The SGAM}


The authors of \cite{uslar2019applying}, \citeauthor{uslar2019applying} describes an alternative model for the \acrshort{sg}, originating from the \acrfull{sgcg}, 
and visualised in figure \ref{fig:SGAM}. The model consists of three dimensions, named "Domains", "Zones", and the "Interoperability dimension". \\ 

\begin{figure}[t]
\includegraphics[width=\textwidth]{figures/SGAM.png}
\caption[SGAM Smart Grid Model]{SGAM Smart Grid Model , as presented in \cite{uslar2019applying}}
\label{fig:SGAM}
\end{figure}

The "Domains" dimension consists of the following domains:

\begin{table}[t]
    \centering
    \begin{tabular}{|p{2.8cm}|p{7cm}|} \hline
   
   
   \textbf{Domain} & \textbf{Description}  \\ \hline
  \textbf{Generation}   & \\ \hline
\textbf{Transmission}      & \\ \hline
\textbf{Distribution}      & \\ \hline
\textbf{DER}     & \\ \hline
\textbf{Customer Premise } & \\ \hline
        
    \end{tabular}
    \caption{SGAM Domains}
    \label{tab:SGAM-Domains}
\end{table}




The "Interoperability" dimension consists of the following Layers:

\begin{table}[t]
    \centering
    \begin{tabular}{|p{2.8cm}|p{7cm}|} \hline
   
   
   \textbf{Layer} & \textbf{Description}  \\ \hline
  \textbf{Business} & \\ \hline
\textbf{Function}  & \\ \hline
\textbf{Information}  & \\ \hline
\textbf{Communication}  & \\ \hline
\textbf{Component layer} & \\ \hline
        
    \end{tabular}
    \caption{SGAM Layers}
    \label{tab:SGAM-Layers}
\end{table}


\begin{table}[t]
    \centering
    \begin{tabular}{|p{2.8cm}|p{7cm}|} \hline
   
   \textbf{Zones} & \textbf{Description}  \\ \hline
  \textbf{Process}   & \\ \hline
\textbf{Field}      & \\ \hline
\textbf{Station}      & \\ \hline
\textbf{Operation}     & \\ \hline
\textbf{Enterprise} & \\ \hline
\textbf{Market} & \\ \hline
        
    \end{tabular}
    \caption{SGAM Zones}
    \label{tab:SGAM-Zones}
\end{table}








