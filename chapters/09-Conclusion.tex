\chapter{Conclusion}

%You definitely should use the \texttt{ntnuthesis} \LaTeX{} document class for your thesis.

\section{Introduction}
In order to provide answers to  the research questions, I include them once more: 


\begin{enumerate}
    \item Which effects of the time delay attack simulations covered by this study, is observable on the visualised output of the PMU simulated?
    \item For a selected similarity requirement, what delay level could be observed as being within similarity tolerance levels?
    \item Which of the delay functions covered would be preferred, in order for the malicious threat actor to stay undetected?    

    %\item How might a \acrshort{sp} attack be mitigated?
    %\item Investigate the GPS spoofing vulnerability of the \acrshort{sg} monitoring and control system.
    %\item Investigate GPS Spoofing detection and mitigation techniques. 
    %\item Investigate how \acrfull{sdn} might be applicable to improve \acrshort{sg} Security
    %\item Investigate SDN-based SG \acrshort{dos} detection and mitigation potentials. 
    %\item 
\end{enumerate}



\section{Summary}




\section{Considerations related to further work}
\begin{itemize}
    \item One option for further work could be to investigate the options of the Power Source object included in the PowerSource subsystem, possibly redesigning the PowerSource with a more realistic Smart Grid Transmission system.
\item Another option could be to replace the PMUs of the Dual PMU subsytem, with different PMU implementations.
\end{itemize}


Those options are independent, and could be combined.

