

\


\subsubsection{Advanced Persistent Threat (APT)}

The most advanced, sometimes stealthy, Cyber Attacks are imposing an imminent uncertainty amongst those tasked with the safe operation of critical systems like the \acrshort{sg}. Threat Actors capable of breaking in to presumably secure critical systems, staying undetected for several months or years, constitutes a persistent threat to the secure operation of the system.


\
\subsubsection{Attack Types}
As described in \cite{ullmann2009delay}, a number of threats affecting the \acrshort{ntp} and  \acrshort{ptp} time protocols is observed:



\begin{itemize}
    \item Grand Master Time Source Attack
    \item \acrfull{dos} attacks: Specialised protocol-specific \acrshort{dos} attacks, as well as traditional IP-based \acrshort{dos} attacks.
    \item Cryptographic Performance Attacks
    \item Packet Manipulation Attacks
    \item Spoofing Attacks
    \item Replay Attack   
    \item Rouge Master attack 
\end{itemize}


Before focusing on smart grid time delay attacks, an introductory overview of a selection of other cyber attacks targeting  the smart grid will be given.


\cite{itkin2017security} provides information on \acrshort{ptp} delay attacks.


\section{An introduction to Smart Grid Cyber Attacks}
As described in \cite{ullmann2009delay}
\subsection{Put all clocks back}

\begin{equation} \label{DSM}
    D_{prox}=D_{SM}\frac{1+d}{d},d>1 
\end{equation}
\subsection{Put specific clock back}
\begin{equation} \label{DMS}
    D_{prox}=D_{MS}\frac{1+d}{d},d>1 
\end{equation}

\cite{finkenzeller2022feasible}