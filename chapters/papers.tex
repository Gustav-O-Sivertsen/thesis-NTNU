% First paper

\begin{paper}{papers/landes1951scrutiny.pdf}{paper:scrutiny}
    Here, you may add a description of the paper, an illustration, or just give the bibliographic reference:
    \begin{quote}
        \fullcite{landes1951scrutiny}
    \end{quote}
    Or you may leave it empty, if you like.
\end{paper}

% Second paper etc.


\subsection{Cyber Attack models}

The complexity of \acrshort{ics} systems introduces a number of challenges to anyone tasked with their protection, as well as to anyone with the intent of targeting the systems with a high impact, but stealthy  cyber attack.


In order to visualise the increased complexity of \acrshort{ics}s, two attack models are covered, the Cyber Kill Chain and the \acrlong{ics} Cyber Kill Chain, where the latter is based on the former.

\subsubsection{The Cyber kill chain}
The Cyber Kill Chain is a model designed for the analysis of attacks on common Information Technology infrastructure. The model is an adoption of the military kill chain model developed by armed forces, based on prior military experience.
\begin{enumerate}
    \item \textbf{Reconnaissance} Information gathering on the target, utilising publicly available resources.
    \item \textbf{Weaponization} Preparation of malicious payload off target premises.
    \item \textbf{Delivery} Intruding the premises of the target, with malicious intents.  
    \item \textbf{Exploitation} Any activity related to information gathering on th premises of the target.
    \item \textbf{Installation} of any malicious software relevant for the attack
    \item \textbf{Command and control} A means by which the attacker establishes a control channel, enabling further actions.
    \item \textbf{Action on objectives} relative to the intents of the attack: Stealing information, controlling equipment, causing service disruption or termination.
\end{enumerate}
Given the complexity of \acrlong{ics}s, consisting of internal security barriers protecting critical system elements, it has not proven to be optimal for analysing attacks on the \acrshort{sg}.\\ 

Following the introduction of the Cyber Kill Chain, the model has been extended to be more suitable for the analysis of more specialised systems, like the \acrshort{sg}.
For the purpose of this thesis, the \acrfull{ics} Cyber Kill Chain, is described.


\subsubsection{The ICS Cyber kill chain}

\acrshort{ics} systems is, nowadays, more often then not, reachable through a common Information Technology Infrastructure, which could be attacked following the Cyber Kill Chain model.

Therefore, the \acrshort{ics} Cyber Kill Chain Model contains two stages, where the first stage, similar to the cyeber Kill chain, has the goal of entering the local area network providing access to the internal network containing the specialised system being the original target of the attack. For the second phase, the attacker typically has operator-like access to the internal network. From internal network, the threat actor has the potential ability of learning the internals of the internal network, preparing for the ulitmate attack on the targeted system.





\subsubsection{The CIA triad  as targets of security breaches}
As described in \cite{rawat2015cyber}, the Cyber Attacks aims to breach the security of the \acrlong{sg}, by targeting any of the areas of  the \acrfull{cia} triad of information security:

\begin{itemize}
    \item Breaching the \acrshort{sg} \textbf{Confidentiality} implies malicious users might take advantage of getting access to unauthorised information, possibly using\footnote{Like stealing from people absent from homes, as indicated by low demand for electricity.} the information in order to harm  others.
    \item Breaching the \acrshort{sg} \textbf{Integrity} implies unintended data alterations, or incidents of malicious actors modifying the proper data exchange of the \acrshort{sg}, by deliberate alteration, removal, or insertion, of vital information contained within \acrfull{sg}. 

 \item Breaching the \acrshort{sg} \textbf{Availability} implies occurrences of unintended service outages, or incidents of malicious actors deliberately attacking the infrastructure by inflicting a \acrfull{dos} attack on the infrastructure.
\end{itemize}


\subsubsection{Traditional networking}
Traditional enterprise-class computer networks consists of a number of proprietary black-boxes, like branded\footnote{For instance Cisco, Catalyst and Juniper} routers and switches, consisting of data forwarding hardware network ports,controlled by a proprietary operating system. 

\begin{itemize}
\item{Vulnerable to (D)DoS} Traditional networks are vulnerable to \acrfull{ddos} attacks. \\

\item{Hard to manage} As each brand of network devices has its own proprietary management software, the flexibilty 
\end{itemize}

\begin{itemize}
\item cyber vulnerability assessment
\item redundant controllers

\item DDoS mitigation survey \cite{hameed2018sdn}
\item compares SDN controllers, recommending OpenDaylight\cite{arbettu2016security}.  \fullcite{arbettu2016security} 


\item traffic load balancing\cite{ejaz2019traffic} \\ \fullcite{ejaz2019traffic}

\end{itemize}



\subsubsection{Network Function Virtualisation}

Network Function Virtualisation (NFV) 




As part of the process of implementing the \acrshort{sg}, a thorough analysis of the security of the underlying network infrastructure is mandatory. 
In \cite{Shapsough2015},Shapsough et. al. discusses the cyber security of the \acrshort{sg}, related to various security requirements, including availability.The consequence of a successful \acrshort{dos} attack is a loss of electricity, which could inflict a serious impact on the modern society.

 %\fullcite{Shapsough2015}

%\acrfull{dos} attacks are identified as the most common threat to the \acrshort{sg}.


\begin{table}[ht]
\centering
\begin{tabular}{|c|l| p{8.5cm}| }
\hline
&Domain &Roles/Services in the Domain \\ \hline
 1&Customer &The end users of electricity. May also generate, store, and manage the
use of energy. Traditionally, three customer types are discussed, each
with its own domain: residential, commercial, and industrial. \\ \hline
 2&Markets&The operators and participants in electricity markets. \\\hline
 3&Service Provider &The organizations providing services to electrical customers and to
 utilities. \\\hline
 4&Operations & The managers of the movement of electricity. \\ \hline
 5&Generation &The generators of electricity. May also store energy for later
distribution. This domain includes traditional generation sources
(traditionally referred to as generation) and distributed energy
resources (DER). At a logical level, “generation” includes coal,
nuclear, and large-scale hydro generation usually attached to
transmission. DER (at a logical level) is associated with customer-
and distribution-domain-provided generation and storage, and with
service-provider-aggregated energy resources.\\ \hline
 6 & Transmission & The carriers of bulk electricity over long distances. May also store
and generate electricity. \\ \hline
 7 &Distribution &The distributors of electricity to and from customers. May also store
and generate electricity. \\
\hline
\end{tabular}
\caption{Table 5-1. Domains and Roles/Services in the\acrlong{sg} Conceptual Model}
\label{tab:SmartGRID-Roles-of-domains}
\end{table}







%The \acrshort{scada} system of the traditional \acrshort{sg} is expanded into the \acrfull{wams} system.


