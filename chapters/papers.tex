% First paper

\begin{paper}{papers/landes1951scrutiny.pdf}{paper:scrutiny}
    Here, you may add a description of the paper, an illustration, or just give the bibliographic reference:
    \begin{quote}
        \fullcite{landes1951scrutiny}
    \end{quote}
    Or you may leave it empty, if you like.
\end{paper}

% Second paper etc.




\subsubsection{Traditional networking}
Traditional enterprise-class computer networks consists of a number of proprietary black-boxes, like branded\footnote{For instance Cisco, Catalyst and Juniper} routers and switches, consisting of data forwarding hardware network ports,controlled by a proprietary operating system. 

\begin{itemize}
\item{Vulnerable to (D)DoS} Traditional networks are vulnerable to \acrfull{ddos} attacks. \\

\item{Hard to manage} As each brand of network devices has its own proprietary management software, the flexibilty 
\end{itemize}

\begin{itemize}
\item cyber vulnerability assessment
\item redundant controllers

\item DDoS mitigation survey \cite{hameed2018sdn}
\item compares SDN controllers, recommending OpenDaylight\cite{arbettu2016security}.  \fullcite{arbettu2016security} 


\item traffic load balancing\cite{ejaz2019traffic} \\ \fullcite{ejaz2019traffic}

\end{itemize}



\subsubsection{Network Function Virtualisation}

Network Function Virtualisation (NFV) 




As part of the process of implementing the \acrshort{sg}, a thorough analysis of the security of the underlying network infrastructure is mandatory. 
In \cite{Shapsough2015},Shapsough et. al. discusses the cyber security of the \acrshort{sg}, related to various security requirements, including availability.The consequence of a successful \acrshort{dos} attack is a loss of electricity, which could inflict a serious impact on the modern society.

 %\fullcite{Shapsough2015}

%\acrfull{dos} attacks are identified as the most common threat to the \acrshort{sg}.


\begin{table}[ht]
\centering
\begin{tabular}{|c|l| p{8.5cm}| }
\hline
&Domain &Roles/Services in the Domain \\ \hline
 1&Customer &The end users of electricity. May also generate, store, and manage the
use of energy. Traditionally, three customer types are discussed, each
with its own domain: residential, commercial, and industrial. \\ \hline
 2&Markets&The operators and participants in electricity markets. \\\hline
 3&Service Provider &The organizations providing services to electrical customers and to
 utilities. \\\hline
 4&Operations & The managers of the movement of electricity. \\ \hline
 5&Generation &The generators of electricity. May also store energy for later
distribution. This domain includes traditional generation sources
(traditionally referred to as generation) and distributed energy
resources (DER). At a logical level, “generation” includes coal,
nuclear, and large-scale hydro generation usually attached to
transmission. DER (at a logical level) is associated with customer-
and distribution-domain-provided generation and storage, and with
service-provider-aggregated energy resources.\\ \hline
 6 & Transmission & The carriers of bulk electricity over long distances. May also store
and generate electricity. \\ \hline
 7 &Distribution &The distributors of electricity to and from customers. May also store
and generate electricity. \\
\hline
\end{tabular}
\caption{Table 5-1. Domains and Roles/Services in the\acrlong{sg} Conceptual Model}
\label{tab:SmartGRID-Roles-of-domains}
\end{table}







%The \acrshort{scada} system of the traditional \acrshort{sg} is expanded into the \acrfull{wams} system.


